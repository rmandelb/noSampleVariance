\documentclass[preprint]{aastex}

%=====================================================================
% CUSTOM: PACKAGES, MACROS & SETTINGS
%=====================================================================
\usepackage[textsize=small]{todonotes} % For comments on things to code up
\newcommand{\wpone}[1]{\todo[inline, color=orange!40]{#1}}
\newcommand{\wptwo}[1]{\todo[inline, color=green!40]{#1}}
\newcommand{\wpthree}[1]{\todo[inline, color=blue!40]{#1}}
\newcommand{\wpfour}[1]{\todo[inline, color=yellow!40]{#1}}

% packages for figures
\usepackage{graphicx}
% packages for symbols
\usepackage{latexsym,amssymb}
% AMS-LaTeX package for e.g. subequations
\usepackage{amsmath}
% Bold greek symbols
\usepackage{bm}

% \newcommand{\begin{equation}}{\begin{equation}} 
% \newcommand{\end{equation}}{\end{equation}}
\newcommand{\half}{\frac{1}{2}}
\newcommand{\hmsun}{\ensuremath{h^{-1}M_{\odot}}}
\newcommand{\hgpc}{$h^{-1}$Gpc}
\newcommand{\hmpc}{$h^{-1}$Mpc}
\newcommand{\hkpc}{$h^{-1}$kpc}
\newcommand{\kv}{\mathbf{k}}
\newcommand{\xv}{\mathbf{x}}
\newcommand{\phiv}{\vec{\phi}}
% \newcommand{\begin{equation}}{\begin{equation}}
% \newcommand{\end{equation}}{\end{equation}}

%%% Data vector
\newcommand{\data}{\mathbf{d}}
%%% 'External' data vector
\newcommand{\dataext}{\data_{\rm ext}}
%%% Angular positions on the sky
\newcommand{\skyangle}{\phi}
%%% Initial conditions of the cosmological mass density
\newcommand{\rhoic}{\rho^{\rm IC}}
%%% Normal distribution
\newcommand{\normdist}{\mathcal{N}}
%%% Mean number density
\newcommand{\nbar}{\bar{n}}
%%% Mass threshold for cluster number counts
\newcommand{\mth}{M_{\rm th}}
%%% Signal covariance matrix
\newcommand{\smat}{\mathsf{S}}
%%% Noise covariance matrix
\newcommand{\noisemat}{\mathsf{N}}
%%% Generic covariance matrix
\newcommand{\Sigmamat}{\mathsf{\Sigma}}
%%% Transfer function relating the mass density to the data vector
\newcommand{\transfer}{\mathsf{T}}
%%% Inverse transfer function operator
\newcommand{\invtransfer}{\mathsf{R}}
%%% Parameter vector of delta_b 
\newcommand{\deltabvec}{\bm \delta_{b}}


%=====================================================================
% FRONT MATTER
%=====================================================================

\slugcomment{Draft \today}

%=====================================================================
% BEGIN DOCUMENT
%=====================================================================
\begin{document}

\title{Reducing sample variance in covariances}

\author{A bunch of DESC hackers}
\begin{abstract}
Most cosmological parameter inferences are now limited by sample variance rather than 
observational measurement errors. \citet{takada2013} showed that the dominant contribution 
to the sample variance of two-point statistics of cosmological density perturbations is from 
fluctuations in the mean density in observed survey volumes. 
We show that the mean density perturbation in a given volume can be constrained with a combination of 
cosmological probes to both reduce the sample variance uncertainties and add additional 
cosmological information about super-survey density perturbations.
\end{abstract}



% -------------------------------------------------------------------
\section{Joint probes with common background density fluctuation} % (fold)
\label{sec:joint_probes}

Consider a measurement of the cosmic shear angular power spectrum,
\begin{equation}
	%\data = \hat{\xi}_+(\phi)
	\data \equiv \hat{C}_{\ell},
\end{equation}
for multipoles $\ell$.
We will consider how the background density contributing to the sample variance 
uncertainties in $\hat{C}_{\ell}$ can be constrained by the 
addition of cluster number counts in the same survey volume,
\begin{equation}
	\dataext \equiv \hat{n}_\text{clust}(M>M_\text{min} | \rho^\text{IC}) =
	\int_{M_\text{min}}^{\infty} n(M)\,\mathrm{d}M.
\end{equation}

Assuming the dominant measurement error on the cosmic shear power spectrum is from shape noise,
the likelihood function is a multivariate Gaussian distribution (denoted by $\normdist({\rm mean}, {\rm covariance})$),
\begin{equation}
	P(\data|\rho^\text{IC},\theta) = \normdist(C_{\ell}(\rho^\text{IC},\theta), \Sigma_{\gamma}),
\end{equation}
where we have included explicit dependence on the mass density initial conditions $\rho^\text{IC}$ 
and cosmological parameters $\theta$.
Here it's important to keep in mind that this is not $\bar{C}_{\ell}$, the expected average correlation
function for a given cosmology.  It is the specific $C_{\ell}$ {\em for the given set of ICs}. The
covariance $\Sigma_{\gamma}$ is just shape noise.  {\it For simplicity we might consider halo model
approximations for the 1-halo and 2-halo terms for the cosmic shear as written in the Mo et al. book
that Sukhdeep found.}

Likewise,
\begin{equation}
	P(\dataext|\rho^\text{IC},\theta) \approx \normdist(n_\text{clust}(M>M_\text{min} |
	\rho^\text{IC}), \smat(\theta)),
\end{equation}
where the covariance is determined by the linear power spectrum and we assume the number of clusters 
is large enough that the Poisson statistics can be approximated by a Gaussian distribution.

Then we can write
\begin{equation}
	P(\data|\rho^\text{IC},\theta)P(\dataext|\rho^\text{IC},\theta)P(\rho^\text{IC}|\theta)
	= 
	P(\rho^\text{IC}, \data,\dataext| \theta)
\end{equation}
\begin{equation}
	\int \mathrm{d}\rho^\text{IC} P(\rho^\text{IC}, \data,\dataext |\theta) 
	= P(\data, \dataext | \theta).
\end{equation}


% -------------------------------------------------------------------
\subsection{Cosmic shear likelihood}

\begin{equation}
- \ln L = \frac{1}{2}\; \sum_{i,j=1}^{N_\ell} \left(  \hat{C}(\ell_i) - C_{\rm model}(\ell_i) \right)\; {\rm Cov}^{-1}\{  C(\ell_i), C(\ell_j) \}\; \left(  \hat{C}(\ell_j) - C_{\rm model}(\ell_j) \right)\;,
\end{equation}
where $N_\ell$ is the number of angular frequency bins.

\begin{equation}
{\rm Cov}\{  C(\ell_i), C(\ell_j) \}= \delta_{ij}\; \frac{4 \pi}{A_{\rm survey} \ell_i \Delta \ell}\; \left( C(\ell_i) + \frac{\sigma_\epsilon^2}{2 \bar{n}} \right)^2\;,
\end{equation}
where $\sigma_\epsilon$ is the dispersion if the complex ellipticity, and $\bar{n}$ the mean number density of the source galaxy sample.


% -------------------------------------------------------------------
\subsection{Halo mass function}
\citet{lima2004} present the probability distribution of the cluster number counts.

% -------------------------------------------------------------------
\subsection{Dependence on mass density phases} % (fold)
\label{sub:dependence_on_mass_density_phases}
We need to calculate the shear two-point function $\xi(\skyangle, \rhoic)$, 
the halo mass function $n(m, \rhoic)$
and the halo bias $b(m, \rhoic)$ as functions of the initial conditions of the 
cosmological mass density $\rhoic$. 
Because the traditional two-point function estimator averages over the phases 
of the mass density, we primarily care about the phases of the super-survey 
modes. This is the `super-sample covariance' discussed by 
\citet{takada2013}.

% -------------------------------------------------------------------
\subsubsection{Power spectrum fluctuations}
We're really interested in how the 3D mass density fluctuations change the
shear correlation function, the halo mass function, \emph{and their 
cross-covariances}. But, we can make a reasonable simplifying assumption that 
the effect of un-marginalized mass density fluctuations is to change the mean density 
in an observed survey volume. 

\citet{takada2013} introduce the Taylor expansion,
\begin{equation}\label{eq:power_perturbation}
	P(k) \rightarrow P(k) \left(
	1 + \frac{\partial \ln P(k)}{\partial \delta_b} \delta_b\right),
\end{equation}
for a `background' density fluctuation $\delta_b$.

For the 2-halo term~\citep[eq. 39 of][]{takada2013} the response of the power spectrum 
to a background fluctuation is approximately a constant, 
\begin{equation}
	\frac{\partial \ln P(k)}{\partial \delta_b} \approx 
	\frac{68}{21}.
\end{equation}
\wpthree{WP3: modify the matter power spectrum with $\delta_b$ value and implement power spectrum derivative with respect to $\delta_b$.}

We can define the background fluctuation as an integral of the 
mass density perturbation over the survey window $W^{(i)}_{\alpha}$ for  
probe $\alpha$ (e.g., shear or cluster counts) and tomographic bin $i$,
\begin{equation}
	\delta_{b,\alpha}^{(i)} \equiv \frac{1}{V_{W_{\alpha}^{(i)}}}
	\int d^{2}\phiv \int d\chi\,
	\int \frac{d^{3}k}{(2\pi)^3}\,
	e^{i \kv\cdot(\chi\phiv, \chi)}
	W^{(i)}_{\alpha}(\kv, \chi) \delta(\kv, \chi),
\end{equation}
where $\chi$ is the comoving line-of-sight distance and $\phiv$ is the 
angular position on the sky. The $\chi$ arguments in the window function 
and density perturbation account for evolution over the observed light-cone.

If the survey volume is large enough so that the 
largest density modes that fit in the survey are in the linear regime, 
then $\delta_{b,\alpha}$ should be approximately Gaussian distributed 
with zero mean and covariance,
\begin{align}\label{eq:delta_b_cov}
	\left<\delta_{b,\alpha}^{(i)} \delta_{b,\beta}^{(j)}\right> &=
	\frac{A}{4V_{W_{\alpha}^{(i)}} V_{W_{\beta}^{(j)}}}
	\int d\chi\, 
	\int \frac{d^{3}k}{(2\pi)^3}\,
	W^{(i)}_{\alpha}(\kv, \chi) W^{(j)}_{\beta}(\kv,\chi) P^{\rm lin}(\kv,\chi) 
	\int_{0}^{\eta_{\rm max}} \eta\, d\eta\, J_{0}(k\chi\eta)
\end{align}
where we made the standard Limber integral approximations to collapse the 
integrals over separate line-of-sight distances, $A$ is the total sky area of the 
two surveys, and $\eta_{\rm max}$ is the maximum angular scale probed by the 
overlap of the two surveys in the sky.
{\bf TODO: double check the prefactors in this expression.}

We might recognize an angular correlation function from Equation~\ref{eq:delta_b_cov},
\begin{equation}
	w_{\alpha\beta}^{(i,j)}(\eta) \equiv 
	\int d\chi\, \int \frac{d^3k}{(2\pi)^3}
	W^{(i)}_{\alpha}(\kv, \chi) W^{(j)}_{\beta}(\kv,\chi) P^{\rm lin}(\kv,\chi) J_{0}(k\chi\eta).
\end{equation}
Although the use of the survey window functions here is  different 
from the standard angular correlation function definitions.
Then,
\begin{equation}
	\left<\delta_{b,\alpha}^{(i)} \delta_{b,\beta}^{(j)}\right> =
	\frac{A}{4V_{W_{\alpha}^{(i)}} V_{W_{\beta}^{(j)}}}
	\int_{0}^{\eta_{\rm max}} \eta\,d\eta\, w^{(i,j)}_{\alpha\beta}(\eta).
\end{equation}
A simple approximation to the covariance of $\delta_b$ might then be 
the integral of the angular correlation function derived from the linear 
power spectrum.



\subsubsection{Cluster counts fluctuations}
The number density of clusters of a given mass $m$ and redshift $z$ over the 
sky can be approximated as,
\begin{equation}
	n_{h}(\xv, m, z) \approx \nbar_{h}(m, z)
	\left[1 + b_{h}(m,z) \delta(\xv, z)\right],
\end{equation}
where $b_{h}(m,z)$ is the linear halo bias and
$\delta(\xv,z)$ is the mass overdensity.

We will use observations of the number of clusters above a 
specified mass threshold $\mth$\wptwo{WP2: Calculate number counts given mass function and $\delta_b$},
\begin{equation}
	\bar{N}_{\alpha}^{(i)}(z) \equiv
	\int_{\mth}^{\infty} dm\, 
	\nbar_{h}(m,z)
	\left[
	1 + b_{h}(m,z)
	\int d^{3}x\, W_{\alpha}^{(i)}(\xv) \delta(\xv)
	\right]
\end{equation}

% subsection dependence_on_mass_density_phases (end)

% section joint_probes_with_common_background_density_fluctuation (end)

\section{Analysis} % (fold)
\label{sec:analysis}
When inferring cosmological parameters from our joint probes we need to marginalize 
$\delta_{b,\alpha}^{(i)}$ (for all $i$ and $\alpha$). 
To account for common density modes among different probes in (partially) overlapping volumes, 
we can append all $\delta_{b,\alpha}^{(i)}$ parameters to the list of cosmological parameters 
and do a joint inference. By drawing joint posterior samples with MCMC, we can obtain marginal 
inferences on the cosmological parameters by simply discarding the posterior samples of $\delta_{b,\alpha}^{(i)}$.

Equation~\ref{eq:delta_b_cov} shows that the expected variance of $\delta_{b,\alpha}^{(i)}$ 
depends on the matter power spectrum, and thus cosmology. Also, because the mass density perturbations 
are Gaussian distributed on large scales, we are justified in placing a 
multivariate Gaussian prior\wpone{WP1 CosmoSIS module: Prior on $\delta_b$ parameter vector. 1) Compute covariance from power spectrum and 2) evaluate ln-prior for sampling.} on 
the parameter vector,
\begin{equation}
	\deltabvec \equiv \left\{\delta_{b,\alpha}^{(i)}; \alpha=1,\dots,N_{\rm probes}, 
	i=1,\dots,N_{\rm z-bins}\right\},
\end{equation}
\begin{equation}
	P(\deltabvec | \theta) = \normdist \left(0, \Sigmamat_{\delta_b}(\theta)\right),
\end{equation}
where the elements of $\Sigmamat_{\delta_b}$ are defined by Equation~\ref{eq:delta_b_cov}.
Crucially, we include the cross-covariances of all $\delta_b$ parameters as determined from 
the Fourier-space overlaps of the survey window functions.

% section analysis (end)

\bibliographystyle{apj}
\bibliography{noSampleVariance}


\appendix
% -------------------------------------------------------------------
\section{Motivation from the hack day}

Why does overlapping volume in two surveys potentially help us reduce sample variance?
This section describes our line of thinking from the 2014-02-06 hack day. 
But, much of this derivation may ultimately be only loosely related to the examples
we calculate in Section~\ref{sec:joint_probes}.

% -------------------------------------------------------------------
\subsection{Basic framework}
Quite generally, we can think of the likelihood of some cosmological probe data $\data$ given cosmological 
parameters $\theta$ as a marginalization over the particular mass density realizations in the universe $\rho$.
That is, we usually only constrain cosmology from statistics such as the power spectrum or correlation 
function that average over the phases of the mass density perturbations. This is because, under the model 
of inflation, mass density perturbations are random fluctuations with a power spectrum (only) that depends 
on the cosmological model.

We can write the marginal likelihood as,
\begin{equation}
	P(\data|\theta) = \int \mathrm{d}\rho^\text{IC}\int \mathrm{d}\rho(t) P(\data|\rho(t),\theta)
	P(\rho(t)|\rho^\text{IC},\theta) P(\rho^\text{IC}|\theta).
\end{equation}
Here $\data$ is the data vector (e.g., the galaxy distribution and its properties, perhaps encoded
as cosmic shear correlation function or the galaxy 2-point
correlation function), $\theta$ are the cosmological parameters, $\rho^\text{IC}$ are the initial
conditions ($\rho^\text{IC}=\rho(t=\text{early})$), and $\rho(t)$ is the density at late times where we measure cosmic
shear.  All $\rho$ are also implicitly a function of position. Strictly speaking, all $\rho$ are to be understood as the values of three-dimensional random fields at every point in space. If these fields are evaluated at $M$ points, these probability distributions can be approximated as $M$-dimensional distributions of $\rho$ values. 
$P(\rho^\text{IC}|\theta)$ is a Gaussian random variable with mean $0$, which we can write as $N(0,S(\theta))$.  The key point is that
$P(\rho(t)|\rho^\text{IC},\theta)$ is purely deterministic (ignoring baryonic physics) - it's just
gravity acting on the matter distributed in a universe that is growing in a way determined by the
cosmological parameters.  So really it's a delta function,
\begin{equation}
	P(\rho(t)|\rho^\text{IC},\theta) = \delta_D(\rho(t)-G(\rho^\text{IC}))
\end{equation} 
where we use $G$ to indicate the actions of gravity.  We use the delta function to eliminate one
integral, over $\rho(t)$, and get
\begin{equation}
	P(\data|\theta) = \int \mathrm{d}\rho^\text{IC}  P(\data|\rho^\text{IC},\theta)N(0,S(\theta)).
\end{equation}
The basic idea is then to get a smaller variance by using some external data, such as massive
cluster counts, to get some idea of the ICs for the survey region.  In that case, we multiply by an
additional $P(\dataext|\rho^\text{IC},\theta)$.  We then get
\begin{equation}
	P(\data,\dataext|\rho^\text{IC},\theta) = P(\data|\rho^\text{IC},\theta)P(\dataext|\rho^\text{IC},\theta).
\end{equation}
The actual signals (the observed $\data$ and $\dataext$) should ideally be highly
correlated, but their {\em uncertainties} should not.  For example, if the external info are counts
of high mass clusters, then the noise comes from Poisson error on X-ray counts or something like
that.  This is unrelated to shape noise in cosmic shear.


Let's say we want to evaluate the integral using Monte Carlo integration, where we carry out the
Monte Carlo integration using $N$ samples (in practice from $N$ simulations).  Without using that
external information, we can write the uncertainty on $P(\data|\theta)$ determined using $N$
simulations as
\begin{equation}
	\text{Error}(P(\data|\theta)) \sim \frac{\text{Var}(P(\data|\rho^\text{IC},\theta))}{N}.
\end{equation}
Using that external information, it becomes
\begin{equation}
	\text{Error}(P(\data|\theta)) \sim \frac{\text{Var}(P(\data,\dataext|\rho^\text{IC},\theta))}{N}.
\end{equation}
So if the variance is significantly smaller, then we might achieve a similar error on our
covariances with fewer simulations.  Note that this (reducing the error of Monte Carlo integration)
is not the goal of this project, really; we actually want to reduce the variance on our cosmological
model parameters in a real data analysis.  We're just using this formalism to get a rough estimate
on how much external data might help us.

So to summarize, if we can evaluate something like
\begin{equation}
	\frac{\text{Var}(P(\data,\dataext|\rho^\text{IC},\theta))}{\text{Var}(P(\data|\rho^\text{IC},\theta))}
\end{equation}
then that tells us the fractional increase in the number of simulations needed for covariance
estimation.  i.e., if this ratio is $1$ then the extra information doesn't help, if this ratio is
$0.5$ then we need half as many sims, and so on.

The corresponding posteriors read
\begin{align}
P(\theta | \vec{d},d_\text{external},\rho^\text{IC}) &\propto P(\vec{d},d_\text{external}|\rho^\text{IC},\theta)\; P(\theta)\\
P(\theta | \vec{d},\rho^\text{IC}) &\propto P(\vec{d}|\rho^\text{IC},\theta)\; P(\theta)\;,
\end{align}
where $P(\theta)$ is the prior. Another useful figure of merit is
\begin{equation}
\frac{\text{Var}(P(\theta_i | \vec{d},d_\text{external}|\rho^\text{IC}))}{\text{Var}(P(\theta_i | \vec{d}|\rho^\text{IC}))}
\end{equation}
for any parameter $i$. A generalised metric could be based on the ratio of the determinants of the parameter covariances.


\end{document}
% 