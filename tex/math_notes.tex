\documentclass[preprint]{aastex}

%=====================================================================
% CUSTOM: PACKAGES, MACROS & SETTINGS
%=====================================================================

% packages for figures
\usepackage{graphicx}
% packages for symbols
\usepackage{latexsym,amssymb}
% AMS-LaTeX package for e.g. subequations
\usepackage{amsmath}

\newcommand{\beq}{\begin{equation}}
\newcommand{\eeq}{\end{equation}}
\newcommand{\hmsun}{\ensuremath{h^{-1}M_{\odot}}}
\newcommand{\hgpc}{$h^{-1}$Gpc}
\newcommand{\hmpc}{$h^{-1}$Mpc}
\newcommand{\hkpc}{$h^{-1}$kpc}
\newcommand{\kv}{\mathbf{k}}
\newcommand{\xv}{\mathbf{x}}
% \newcommand{\beq}{\begin{equation}}
% \newcommand{\eeq}{\end{equation}}

%%% Angular positions on the sky
\newcommand{\skyangle}{\phi}
%%% Initial conditions of the cosmological mass density
\newcommand{\rhoic}{\rho^{\rm IC}}
%%% Normal distribution
\newcommand{\normdist}{\mathcal{N}}
%%% Mean number density
\newcommand{\nbar}{\bar{n}}

%=====================================================================
% FRONT MATTER
%=====================================================================

\slugcomment{Draft \today}

%=====================================================================
% BEGIN DOCUMENT
%=====================================================================
\begin{document}

\title{Reducing sample variance in covariances}

\author{A bunch of DESC hackers}
\begin{abstract}
Something or other.
\end{abstract}

\section{Introduction}

Sarah will fill this in.  Rely on picture of first board and top of second board as well as our discussions.

\section{Math}

Basic framework:
\beq
P(\vec{d}|\theta) = \int \mathrm{d}\rho^\text{IC}\int \mathrm{d}\rho(t) P(\vec{d}|\rho(t),\theta)
P(\rho(t)|\rho^\text{IC},\theta) P(\rho^\text{IC}|\theta).
\eeq
Here $\vec{d}$ is the data vector (e.g., the galaxy distribution and its properties, perhaps encoded
as cosmic shear correlation function or the galaxy 2-point
correlation function), $\theta$ are the cosmological parameters, $\rho^\text{IC}$ are the initial
conditions ($\rho^\text{IC}=\rho(t=\text{early})$), and $\rho(t)$ is the density at late times where we measure cosmic
shear.  All $\rho$ are also implicitly a function of position.
$P(\rho^\text{IC}|\theta)$ is a Gaussian random variable with mean $0$, which we can write as $N(0,S(\theta))$.  The key point is that
$P(\rho(t)|\rho^\text{IC},\theta)$ is purely deterministic (ignoring baryonic physics) - it's just
gravity acting on the matter distributed in a universe that is growing in a way determined by the
cosmological parameters.  So really it's a delta function,
\beq
P(\rho(t)|\rho^\text{IC},\theta) = \delta_D(\rho(t)-G(\rho^\text{IC}))
\eeq 
where we use $G$ to indicate the actions of gravity.  We use the delta function to eliminate one
integral, over $\rho(t)$, and get
\beq
P(\vec{d}|\theta) = \int \mathrm{d}\rho^\text{IC}  P(\vec{d}|\rho^\text{IC},\theta)N(0,S(\theta)).
\eeq
The basic idea is then to get a smaller variance by using some external data, such as massive
cluster counts, to get some idea of the ICs for the survey region.  In that case, we multiply by an
additional $P(d_\text{external}|\rho^\text{IC},\theta)$.  We then get
\beq
P(\vec{d},d_\text{external}|\rho^\text{IC},\theta) = P(\vec{d}|\rho^\text{IC},\theta)P(d_\text{external}|\rho^\text{IC},\theta).
\eeq
The actual signals (the observed $\vec{d}$ and $d_\text{external}$) should ideally be highly
correlated, but their {\em uncertainties} should not.  For example, if the external info are counts
of high mass clusters, then the noise comes from Poisson error on X-ray counts or something like
that.  This is unrelated to shape noise in cosmic shear.


Without using that external information, we can write the uncertainty on $P(\vec{d}|\theta)$
determined using $N$ simulations as
\beq
\text{Error}(P(\vec{d}|\theta)) \sim \frac{\text{Var}(P(\vec{d}|\rho^\text{IC},\theta))}{N}.
\eeq
Using that external information, it becomes
\beq
\text{Error}(P(\vec{d}|\theta)) \sim \frac{\text{Var}(P(\vec{d},d_\text{external}|\rho^\text{IC},\theta))}{N}.
\eeq
So if the variance is significantly smaller, then we might achieve a similar error on our
covariances with fewer simulations.



Second board

Specific case: external data is cluster counts.  Observed data is shear correlation function.  Need
1h and 2h in principle but we might limit ourselves to one or the other as is convenient.

Equations we wrote for each component.


\subsection{Halo mass function}
\citet{lima2004} present the probability distribution of the cluster number counts.


\subsection{Dependence on mass density phases} % (fold)
\label{sub:dependence_on_mass_density_phases}
We need to calculate the shear two-point function $\xi(\skyangle, \rhoic)$, 
the halo mass function $n(m, \rhoic)$
and the halo bias $b(m, \rhoic)$ as functions of the initial conditions of the 
cosmological mass density $\rhoic$. 
Because the traditional two-point function estimator averages over the phases 
of the mass density, we primarily care about the phases of the super-survey 
modes. This is the `super-sample covariance' discussed by 
\citet{takada2013}.

\subsubsection{Power spectrum fluctuations}
We're really interested in how the 3D mass density fluctuations change the
shear correlation function, the halo mass function, \emph{and their 
cross-covariances}. But, we can make a reasonable simplifying assumption that 
the effect of un-marginalized mass density fluctuations is to change the mean density 
in an observed survey volume. 
\begin{equation}
	P(k) \rightarrow P(k) \left(
	1 + \frac{\partial \ln P(k)}{\partial \delta_b} \delta_b\right)
\end{equation}

\begin{equation}
	\delta_b \sim \normdist
	\left(0, \left(\sigma_W^L\right)^{2}\right)
\end{equation}
where,
\begin{equation}
	\left(\sigma_{W}^{L}\right)^{2} \equiv
	\frac{1}{V_{W}^{2}} 
	\int \frac{d^{2}\kv}{(2\pi)^3}
	\left|W(\kv)\right|^{2}
	P^{L}(k).
\end{equation}

For the 2-halo term~\citep[eq. 39 of][]{takada2013}, 
\begin{equation}
	\frac{\partial \ln P(k)}{\partial \delta_b} \approx 
	\frac{68}{21}
\end{equation}

\subsubsection{Cluster counts fluctuations}
The number density of clusters of a given mass $m$ and redshift $z$ over the 
sky can be approximated as,
\begin{equation}
	n_{h}(\xv, m, z) \approx \nbar_{h}(m, z)
	\left[1 + b_{h}(m,z) \delta(\xv, z)\right],
\end{equation}
where $b_{h}(m,z)$ is the linear halo bias and
$\delta(\xv,z)$ is the mass overdensity.

% subsection dependence_on_mass_density_phases (end)

\bibliographystyle{apj}
\bibliography{noSampleVariance}

\end{document}
% 